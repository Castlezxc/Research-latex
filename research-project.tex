%% LyX 2.2.2 created this file.  For more info, see http://www.lyx.org/.
%% Do not edit unless you really know what you are doing.
\documentclass[english]{article}
\usepackage[T1]{fontenc}
\usepackage[latin9]{inputenc}
\usepackage[a4paper]{geometry}
%\geometry{verbose,tmargin=3cm,bmargin=2.5cm,lmargin=2.5cm,rmargin=2.5cm}
\usepackage{color}
\definecolor{note_fontcolor}{rgb}{0.800781, 0.800781, 0.800781}
\usepackage{array}
\usepackage{verbatim}
\usepackage{rotfloat}
\usepackage{url}
\usepackage{amstext}
\usepackage{amsthm}
\usepackage{graphicx}
\usepackage{setspace}
\usepackage[backend=biber, style=apa]{biblatex}
%\bibliography{refs}
\addbibresource{refs.bib}
\usepackage{csquotes}
\usepackage{gensymb}
\usepackage{multirow}

%%%%%%%%%%%%%%%%%%%%%%%%%%%%%% LyX specific LaTeX commands.
%% Because html converters don't know tabularnewline
\providecommand{\tabularnewline}{\\}
%% The greyedout annotation environment
\newenvironment{lyxgreyedout}
{\textcolor{note_fontcolor}\bgroup\ignorespaces}
{\ignorespacesafterend\egroup}
%%%%%%%%%%%%%%%%%%%%%%%%%%%%%% User specified LaTeX commands.
\usepackage{amsmath,epsfig,pstricks,pst-text,pst-node,pst-plot,amssymb,wrapfig,threeparttable,rotating}

\makeatother

\usepackage{babel}

\begin{document}
	
	\title{Investigating the links between weather variability and agricultural
		GDP}
	
	\author{Kevin Hendriks 17043779\\ 
		Danae Pavlou 17021856\vspace{0.5cm}\\
		NPN 780 Research Report\vspace{2cm}\\
		Submitted in partial fulfillment of the degree BSc(Hons) Actuarial
		Sciences\vspace{0.5cm}\\
		Supervisor: S Hossain\vspace{2cm}\\
		Department of Actuarial Sciences, University of Pretoria\vspace{0.5cm}\\
		\includegraphics[width=5cm]{UPlogohighres}\vspace{0.5cm}\\
		September 2020}
	
	\date{}
	
	\maketitle
	\newpage{}
	\begin{abstract}
		Short summary of the research proposed. This will be a two to three
		paragraphs long and aims to fully describe the content and contributions
		of the research report.
		
		\newpage{}
	\end{abstract}
	
	\section*{Declaration}
	
	I, \emph{Kevin Ryan Hendriks and Danae Pavlou}, declare that this essay,
	submitted in
	partial fulfilment of the degree \emph{BSc(Hons) Actuarial Sciences} at
	the University of Pretoria, is my own work and has not been previously
	submitted at this or any other tertiary institution. 
	
	\vspace{1cm}
	
	\_\_\_\_\_\_\_\_\_\_\_\_\_\_\_\_\_\_\_\_\_\_\_\_\_\_\_\_\_
	
	\emph{Kevin Ryan Hendriks}
	
	\vspace{1cm}
	
	\_\_\_\_\_\_\_\_\_\_\_\_\_\_\_\_\_\_\_\_\_\_\_\_\_\_\_\_\_
	
	\emph{Danae Pavlou}
	
	\vspace{1cm}
	
	\_\_\_\_\_\_\_\_\_\_\_\_\_\_\_\_\_\_\_\_\_\_\_\_\_\_\_\_\_
	
	\emph{Saqib Hossain}
	
	\vspace{1cm}
	
	\_\_\_\_\_\_\_\_\_\_\_\_\_\_\_\_\_\_\_\_\_\_\_\_\_\_\_\_\_
	
	Date
	
	\newpage{}
	
	\tableofcontents{}
	
	\listoffigures
	
	\listoftables
	
	\newpage{}
	
	\section{Introduction}
	There is a gap in the coverage of climate related financial products around 
	less shocking, yet still devastating, climate related disasters, specifically
	drought. Existing financial products in the climate sphere often deal with
	natural disasters that produce sudden and obvious claim events such as
	earthquakes or cyclones. 
	
	\subsection{How do we address climate change?}
	Before climate change can be addressed through national and international
	policies, the current status of climate change needs to be established. This
	ha been done in many research papers and is a major focus of several international
	organizations. There are two different categories of policies around climate:
	\begin{itemize}
		\item Policies that try to address the changing environment, try to slow down
		/
		reduce climate change. These policies are focused around limiting human
		activities that contribute towards increasing the amount of carbon in the
		atmosphere.
		\item Policies that look to protect communities, businesses and households
		from
		the negative impacts of climate. These look at changes in the environment and
		how people interact with these changes and are likely to be affected by these
		changes. The insurance industry is best suited to impact this side of the
		research question as, by nature, it is a risk-averse, defensive strategy.
	\end{itemize}
	
	The first looks to fight against future climate change where the second looks
	to defend vulnerable people against the current and future impacts of climate
	change. In order to do this you need to establish what the future climate
	trends
	are likely to be, but it is also important to establish the effect these trends
	are likely to have. Fighting an effective battle against climate change
	requires
	both of these approaches to succeed and currently there is a lack of efforts
	with the aim of the second bullet point above.
	
	\newpage
	
	\section{Problem Statement}
	The ideal for governments and financial institutions is to have methods of
	identifying which human activities contribute to climate change and by how
	much, then to be able to identify the climate events that are relevant to an economy
	and communities. Then to extend that to quantify the financial impact on the
	economy and various communities. Finally, to have financial products Investment
	projects should be aimed at preventing the loss of life and livelihoods in the
	case of a natural disaster or are capable of identifying / predicting the
	impacts of non-preventable changes in the climate, such as increasing
	temperatures. These models, policies and financial products are available to
	everyone and thus the climate related risks are spread globally.\\
	
	However currently there are few early warning systems and few financial
	products that focus on climate change itself rather than on natural disasters as risks.
	The focus of funding is on humanitarian aid after a disaster event and not on
	projects that aid in preventing and coping with natural disasters. Existing
	models that attempt to quantify the impacts of climate change on various
	economies are difficult to access and require special expertise in coding,
	economics and statistics to be able to operate, usually requiring teams to
	adequately cover these fields. Due to the complex nature of these models most
	poorer regions of the world do not have access to the expertise or technology
	in order to run the appropriate analysis.\\
	
	In order to try to bring what is currently available and being implemented
	closer to the ideal we need to explore the effects of the less violent natural
	phenomena as an important point that is not sufficiently addressed by the
	insurance or financial sector, currently. To progress in this regard, it is
	important to find a focal point within the many factors of climate change that
	can be analysed in a quantitative way. It is important to make the study as
	relevant as possible to regions of the world that have the most to gain from
	systematic improvements when it comes to handling climate change. This means
	that areas that have little to no climate analysis combined with large
	exposures
	to climate risk should be where the most urgency is directed when planning for
	future climate change mitigation.\\
	
	In order to keep climate analysis as relevant as possible to the real world,
	the
	quantifiable measures that are used to construct financial products addressing
	climate change should not be abstract. These measures should be easily
	translatable into the real economy and therefore comparable to other financial
	risks. You want these quantifiable measures to be used in standard risk
	analyses
	where both the likelihood of a risk event occurring as well as the financial
	impact of a risk event occurring are important in decisions for further action
	by governments and financial institutions. The aim of being understandable and
	relevant to a financial audience as well as comparable to other financial risks
	is to encourage action in this space. 
	
	\newpage
	
	\section{Purpose statement}
	The purpose of this research project is to investigate the relationship between
	agricultural GDP and drought severity in South Africa. 
	
	\begin{itemize}
		\item The global region where the greatest impact from the focus of this study
		is possible is Africa, seeing as water is widely mismanaged and African
		communities are highly exposed to variance in weather conditions. 
		\item South Africa was chosen as a case study for the African continent as it
		has the most data coverage of the variables in this study. 
		\item GDP is chosen as a financially significant measure of the impact of
		climate change. 
		\item Agricultural GDP is chosen as this economic sector is most directly
		influenced by drought events. 
		\item Droughts represent a measurable event that is related to weather
		variability and is the link within the study to climate change.
	\end{itemize}
	
	Water scarcity is the point at which the two focuses of the study meet, those
	being the financial sector and climate change. Water scarcity is defined in
	terms of the supply vs the demand for water. Demand for water is driven by
	economic factors whereas the supply of water is an environmental factor, this
	provides the ideal point of departure for the research project. 
	
	\section{Research objectives}
	\begin{itemize}
		\item Isolate Agricultural GDP from the general GDP statistics 
		\item Collect measurements of drought severity
		\begin{itemize}
			\item Research various measurements of drought severity and pros and cons of
			a
			few different methods.
		\end{itemize}
		\item Identify a measure of water scarcity
		\item Check the collinearity of the measurement of water scarcity and drought
		severity
		\item Perform linear regression on agricultural GDP and the identified
		measurement of drought severity.
	\end{itemize}
	
	\newpage
	
	\section{Literature review}
	
	\subsection{Climate change}
	The literature explores climate change effects on economic growth, finding that
	climate change and higher temperatures will adversely impact GDP in the long
	run. A recurring theme is that the poor, developing countries will feel these
	adverse impacts more severely than affluent countries. It is also found that
	water is the principal means through which the world will feel the effects of
	climate change. According to the \parencite{Water2018} more than 2 million
	people currently live in areas that are experiencing high water stress, and
	this is expected to worsen as the effects of climate change intensify and population
	growth increases the demand for water. It is anticipated that climate change
	will cause hydrological cycles to deepen, causing global changes in
	precipitation patterns and increased extreme events \parencite{Bates2008}.\\
	
	Climate change is made up of environmental factors such as the temperature on
	the Earth, weather patterns around the globe and how these factors are changing
	\parencite{Pielke2004}. Climate change has received international recognition
	as one of the most pressing issues in today?s world. Below is a table showing
	international policy documents and funds that show the global presence of
	climate change:
	
	\begin{table}[ht]
		\centering
		\begin{tabular}{ |l|l|l| }
			\hline
			Publication & Organisation & Published\\
			\hline
			The Sustainable Development Goals & The United Nations & 2015\\
			The Paris Agreement & The International Monetary Fund & 2015\\
			The Stern Review & The Treasury of Great Britain & 2007\\
			\hline
			Fund Name & Type of fund & Established\\
			\hline
			Green Climate Fund & Multilateral fund & 2013\\
			Special Climate Change Fund & Multilateral fund & 2001\\
			Climate Investment Funds & Multilateral fund & 2008\\
			\hline
		\end{tabular}
		\label{table:papers}
	\end{table}
	
	Many of the impacts of climate change are attributed to increases in greenhouse
	gas emissions that lead to rising global temperatures \parencite{Solomon2009}.
	There are several studies that show the negative economic effects of
	temperature increases, as summarised in the table below:\\
	
	\begin{table}[ht]
		\centering
		\begin{tabular}{ |l|l|l|l|}
			\hline
			Author                               & Increase                           & Economic impact      & Region                      \\ \hline
			\cite{Tol2018}                       & 2.5\degree C                       & Income loss of 1.4\% & Global                      \\
			\cite{Farid2016}                     & 3\degree C                         & GDP loss of 2\%      & Global                      \\ \hline
			\multirow{2}{6cm}{\cite{Kompas2018}} & \multirow{2}{1.5cm}{4\degree C}    & GDP loss of 21\%     & South East Asia             \\
			                                     &                                    & GDP loss of 26.6\%   & Africa                      \\ \hline
			\multirow{2}{6cm}{\cite{Roson2012}}  & \multirow{2}{1.5cm}{4.79\degree C} & GDP loss of 12.6\%   & Asia                        \\
			                                     &                                    & GDP loss of 10.3\%   & Middle East \& North Africa \\ \hline
		\end{tabular}
		\label{table:climate_impact}
	\end{table}
	
	\parencite{Tol2018}, who reviewed the 27 published estimates of the total
	economic impact of climate change currently available, also found that the
	developing countries were more severely affected by climate change and that the
	total economic impacts of climate change are negative. However, \parencite{Tol2018} does state that the impact of climate change on economic growth is not well understood and could be too optimistic. The limited 27
	estimates of the total economic impact of climate change make it challenging to
	draw a definitive conclusion.  There are 11 estimates that assume a warming of
	2.5\degree C, 3 estimates are positive while 8 are negative, showing that it is
	unclear whether climate change will lead to a net welfare gain or loss. When
	the 27 estimates are considered as a whole, they indicate that moderate warming
	leads to net positive welfare, while any greater temperature increases lead to
	negative net impacts on welfare. Although the estimates show great uncertainty,
	they suggest that negative impacts are more probable in the long term and that
	if climate change doubles, its impacts more than double, thus making the
	impacts more than linear.\\
	
	\parencite{Tol2018} therefore finds that although the economic impact of
	climate change in the 21st century is modest and not well understood, long-term
	the negative impacts will outweigh the positive ones, with the poor and hot
	developing countries being the most significantly affected by these adverse
	impacts. The Stern Review and the Paris Agreement both highlight that the
	impacts of climate change are poorly understood, but also agree that the
	negative, long-term impacts of climate change will be significant, especially
	in
	developing regions.\\
	
	\parencite{Stern2007} suggests that the distribution and variability of water
	is where people will most significantly feel the impact of climate change. This
	could have substantial impact on growth, production and poverty reduction. The
	water cycle will intensify as climate change alters water availability, and
	manyvareas will see increased severity of droughts and floods. Arid areas, like
	Southern Africa, are likely to have increased water scarcity and
	\parencite{Barrios2010} and \parencite{Brown2011}found that irregular rainfall
	can have a substantial impact on economic growth in sub-Saharan Africa.
	\parencite{Roson2012} also suggests that water scarcity is a major reason for
	the negative welfare effects on the Middle East and North Africa, which are
	expected to be -16.14\% by the end of the century.\\
	
	What has been found is that although the full economic impacts and the extent
	of these impacts is uncertain, most literature agrees that climate change in
	the
	long term will have a negative economic impact, although these impacts will be
	felt differently across the globe. The fundamental finding is that the poor
	will
	suffer the most from the impacts of climate change, specifically through water
	scarcity.
	
	\subsection{Water scarcity}
	A rapidly changing supply of natural water is likely to have a negative effect
	on world development by decreasing household incomes, straining economies that
	have inefficient water usage practices and impacting health, education and
	welfare of developing nations \parencite{WB2016}. \\
	
	
	There is an increasing amount of evidence demonstrating that rainfall shocks
	can have long-term economic effects on human capital. Having no access to clean
	water or no access to water at all has long-term effects on health, human
	capital, and education, especially in young children \parencite{WB2016}.
	\parencite{Almond2011} concurs with this conclusion and mentions that
	households
	facing health risks due to water scarcity are unable to invest in their
	children
	due to reductions in household incomes. A rainfall shock can significantly
	decrease a family's income when income depends on favourable rainfall patterns,
	such as when households rely on rain-fed agriculture to feed their families
	and/or provide an income to satisfy their needs. This lack of income could
	cascade into foetuses and infants developing nutritional deficits which could
	have dire consequences in the long run. This could extend the cycle of poverty,
	as health concerns and malnutrition brought on by floods or droughts can impact
	education and thus future employment \parencite{Damania2017}. Thus, climate
	related water shocks hinder development and increase poverty rates through
	these
	channels.\\
	
	
	The reach of water scarcity and climate change does not stop at the low-income
	portions of a population and can have long-term effects on the health and
	economic potential of rural farmers residing in developing countries as well as
	middle-class workers from middle-income countries. \parencite{Maccini2009}
	conducted a survey of Indonesian rural farmers and found that persons,
	especially women, who in their rest year of life experienced rainfall shocks
	also achieved less education, earned lower incomes and married people who
	earned
	lower incomes. This agrees with the view of the \parencite{WB2016},
	\parencite{Damania2017} and \parencite{Almond2011}. Interestingly, recent
	evidence demonstrates that these results could impact middle-class workers from
	middle-income countries. A study concerning middle class workers in Ecuador
	shows that if rainfall shocks and higher temperature were experienced while
	still in the uterus, the results would see the child experience reduced formal
	sector earnings 20 to 60 years later \parencite{Fishman2015}.\\
	
	
	Both excessive and insufficient rainfall causes GDP growth rates to decrease.
	\parencite{Burke2015} and \parencite{Dell2012} conclude that high temperatures
	have adverse economic impacts. They also identified that rainfall and
	temperature variability are highly correlated and studied the impacts of
	climate
	change and rainfall on the economy. \parencite{Brown2013} used the Weighted
	Anomaly Standardized Precipitation index, which measures rainfall inconsistency
	at extremes, in order to measure water availability. All of these studies
	conclude that increases in temperature and weather variability increase the
	frequency of extreme wet and dry anomalies, which negatively impact economic
	growth, especially in agricultural regions.\\
	
	
	Several studies find that rainfall or water availability has no strong and
	significant effect on economic growth, which is contradicting when considering
	the number of studies that indicate extensive effects of rainfall and water
	availability on economic indicators like GDP.  It has been suggested that a
	reason for the discrepancy of results is due to the statistical approaches
	used,
	which place attention on how water availability or rainfall impacts economic
	growth at the national level. \parencite{Briant2010} showed that this is due to
	spatial averaging over large units, which disguises variability. If a part of a
	country is experiencing floods while the other is experiencing droughts,
	averaging over the country may result in normal rainfall, thus inadequately
	capturing the spatial variation in water availability and the varying effects
	it
	has on a country. A similar argument exists when a study uses data on an
	aggregated time scale such as using annual data. Weather variation within a
	year
	can disguise the extreme weather experiences over each year. However, at
	disaggregated spatial scales the analysis shows that rainfall variability has a
	substantial impact on economic growth \parencite{Damania2019}.\\
	
	
	This evidence agrees with \parencite{Tol2018}, who suggests that many of the
	estimates of the impact of climate change on various economies are optimistic.
	Another study that highlights this underestimation of the impacts of climate
	change and concurs with the current trend of using aggregated climate data is
	\parencite{Mekonnen2016}. They conducted a study that suggests that around 71\%
	of the world experiences water scarcity for at least one month in any given
	year.\\
	
	
	Therefore, the global economy is likely to experience the effect of climate
	change principally through water cycles because of increasing future
	variability
	of extreme natural phenomena such as droughts and floods. 
	
	\subsection{Water management}
	\subsubsection{The role of water management in water scarcity and climate change}
	To protect economic growth as the demand for water increases because of
	growing populations, intensifying water usages and urbanization, so too does
	the
	need for good water management. Unfortunately, throughout the world, unequal
	and
	inefficient water management systems, that absorb large amounts of capital from
	other areas of the economy \parencite{Barbier2004}, and a lack of information
	pose a threat to global development by incapacitating policy responses from
	both
	the public and private sectors \parencite{WB2016} and \parencite{Damania2020}.\\
	
	Water scarcity and climate change bring a wave of problems to most of the
	world and poor water management practices only act to deepen the troughs facing
	policy makers. The Netherlands? Minister of Infrastructure and the Environment,
	Cora van Nieuwenhuizen, stressed the importance of good water management
	saying,
	?After all, 90\% of climate adaptation is about water?.
	\parencite{Cashman2018}.
	
	\subsubsection{Population growth and urbanization and water management}
	Social trends around population growth and urbanization in developing regions demand that adequate water management policies are prepared and implemented	sooner rather than later. Population growth and urbanization increase the proportion of the world?s population that have exposure to natural hazards and disasters as well as adding to the demand for fresh water. This is highlighted by the World Economic Forum, who outline that an increasing global population intensifies the demand for water. \parencite{Mekonnen2016} discusses this and that high levels of water scarcity are often found in areas with high population density. \parencite{CRED2019} highlights the upward trend in urbanization in Africa, which makes up a large part of the developing world and is expected to reach up to 25\% of the global population by 2050. \parencite{UNDRR2015} concludes that resource shortages in urban areas produce circumstances that are	ripe for civil conflict and disease. \parencite{Hall2014} draws a connection between water scarcity and the migration of rural populations to urban areas due	to a loss of livelihoods in agricultural endeavours. \parencite{UNDRR2015} goes on to explain that 60\% of the African population lives in slums and sets this out as a consequence of urbanization. They define a slum as lacking clean water, sanitation facilities, durable housing and sufficient living space. \parencite{WB2016} stresses the fact that most poor populations live in drought prone areas or flood basins, which further increases the likelihood of experiencing a weather phenomenon as well as the potential severity of that event. From an insurance perspective, \parencite{Cashman2018}, highlights that rapid urbanization and increasing population densities in vulnerable areas, combined with increasing standards of living have a significant effect on the level of disaster losses experienced globally. All these conclusions link together to produce a strong indication for the need for water to be efficiently managed, especially in the case of urban water infrastructure.
	
	\subsubsection{The situation on the ground}
	Currently water management policies are inefficient, and models show that if
	they persist water scarcity will worsen where water is already limited and
	become an issue in regions previously unbothered by water scarcity. The
	\parencite{WB2016} predicts that by 2050 some regions could have a 6\% decline
	in the growth rate of GDP caused by water-related losses. The impacts of water
	scarcity and water mismanagement will be felt excessively by the poor, as they
	often rely on rain dependent agriculture, live in drought or flood prone areas,
	and are at risk of poor-quality water and inadequate sanitation. \\
	
	
	\parencite{Barbier2004} looks at 112 developing countries and explores the
	link between GDP growth and water utilisation rates and displays that an
	increase in the use of water tends to slow the growth rates of developing
	countries. \parencite{Barbier2004} states that the reason for this is that an
	increase in water utilisation does not increase productivity gains enough to
	outweigh the greater social and environmental costs associated with the
	infrastructure and organisations needed to achieve greater water security.\\
	
	
	\parencite{WB2016} finds that prudent water management policies can aid in
	protecting economic growth in developing regions. The model they use compares
	two extreme scenarios. SSP1, represents a positive outlook and SSP3 describes a
	world consisting of low adaptation, restricted economic growth and high
	emissions. For each scenario, the results show that water scarcity is a
	noteworthy hindrance to growth and development, however, adverse growth rates
	could worsen if inefficient water-management policies are used. The report also
	highlights that through better water-management, certain regions could see
	their
	GDP growth rate accelerate by 6\%. Thus, judicious water management can
	partially nullify the adverse growth impacts. The greatest water scarcity will
	be faced by the Middle East, Central Asia, parts of South Asia and North
	Africa,
	which corresponds to the results of other studies already discussed. Under the
	?business as usual? scenario, the results of the model suggest that significant
	action would have to be taken in order for climate related water issues to not
	adversely impact global economic growth. \\
	
	
	Water is a resource that is needed directly or indirectly for the production
	of most goods and services, yet any efforts to quantify the contribution of
	water to total economic growth continue to be vague and ambiguous
	\parencite{WB2016}.
	
	\subsubsection{Water Management Practices to respond to intensifying water scarcity issues}
	The water cycle is multidimensional and thus it makes it difficult to address
	issues of water scarcity within a single policy. At the source, water faces
	problems associated with a public good. Water policies are required to protect
	water sources. At the next stage of the water cycle, investment is required to
	convert water into a private good using infrastructure such as pipes and
	reservoirs. However, water is also seen as a human right and thus a merit good
	and so enough water needs to be made available at inexpensive prices. These
	definitions are contradictory and so there is no single price that would
	concurrently be able to meet both objectives \parencite{Damania2020}.\\ 
	
	The distortion of water prices leads to an inefficient and wasteful use of the
	resource, inflicting higher costs on the economy, society, and the environment.
	The marginal value of water for different uses varies substantially, as the
	price paid by residential, industry and agriculture users differ
	\parencite{WB2016}. The price of water rarely reflects its economic value, or
	the costs associated with treating and transporting it. Thus, a price that is
	too low and allows cheap water access may encourage inefficient water use and
	waste, compromising sustainability. Prices that are too high will mean the poor
	will be unable to access a resource that is a basic human right. Therefore,
	water prices must be able to control demand and guarantee unbiased access.\\
	
	An abundance of CGE models show that efficient allocation can lead to
	meaningful benefits. \parencite{Hassan2011} showed that efficient water
	allocation in South Africa could enhance the agricultural GDP by approximately
	4.5\% annually, further \parencite{Roson2017a} show that inefficient allocation
	results in estimated global losses of 6\% of GDP. However, efficient allocation
	can only occur if there is an adequate management and regulation mechanism.
	Technology that is sophisticated enough to monitor, measure and disclose the
	water performance using unbiased metrics has not yet been realised.
	
	\paragraph{Technical efficiency and suitable investment:}
	
	\parencite{Calzadilla2013} discovers that if water efficiency is improved
	there are compound benefits, as water scarce regions benefit, and other regions
	are encouraged to make use of water with more care. It has been shown by CGE
	models that an improvement in the efficient use of water can tackle water
	scarcity issues. One of the ways of encouraging both efficient allocation and
	use of water is through pricing. Alternatively, protection against water shocks
	could be achieved by investment. Developing countries do not have access to
	water storage, flood control and other infrastructure that provides protection
	against water shocks.  Thus, developing countries must be able to more
	accurately predict when natural disaster will occur. This will require
	investing
	in early-warning systems and introducing institutions that have the ability to
	transform a forecast into a warning.\\
	
	\paragraph{Water trade:}
	
	Thanks to globalization, the world can deal with water scarcity through
	trade. Where water is scarce and costly, water intensive goods will be imported
	while goods with low water content will be exported. Therefore, countries that
	have high levels of water scarcity import goods that they would not be able to
	support with local water levels. So water is traded through the water used to
	produce a product. This is known as the ?virtual water trade?, and it is seen
	as
	a way of evading water scarcity in certain regions. However, globalisation has
	meant that production that has high water demands is often transferred to
	developing countries who are already water scarce. It has been seen that water
	scarcity and climate change will more adversely affect developing countries,
	and
	this water intensive production will only compound this
	\parencite{Damania2017}.\\
	
	\paragraph{Increase water supply and availability:}
	
	This could be achieved through investing in projects that increase water
	supply, such as water storage infrastructure like dams, adequate treatment
	plants, water recycling and reuse, desalination projects. However, for these to
	be effective they must be paired with policies that support water efficiency
	and
	efficient water allocation across sectors. An increase in supply must be paired
	with consistent safeguards in order to manage water use. If this is not done,
	demand will rise to match the new supply level resulting in an increased level
	of water dependence, especially in arid regions \parencite{WB2016}.\\
	
	\paragraph{Optimising water use through better planning and incentives:}
	
	Allocating water across sectors to higher-value uses will help achieve
	climate resilient economies that are still able to thrive in uncertain
	climates.
	In order to ensure future water security, water will need to be recognised at a
	scarce and valuable productive resource. This can be realised through economic
	instruments, like water permits and prices, that ensure the delivery of water
	where it is more productive. Water allocations through permits give consumers
	the right to sell or rent the water available to them. A transaction will only
	occur if it is seen as beneficial by both the buyer and the seller, thus
	resulting in a win-win situation for both parties involved. However, this would
	require complex legal structures and must link up with a credible water trading
	system \parencite{WB2016}.\\
	
	Water pricing is generally the easiest, simplest and most efficient way for
	controlling water use, for municipal use. Placing a high price on water could
	provide an incentive to reduce water waste. It could also reduce the municipal
	demand on water. However, with high water prices the most vulnerable will
	struggle to retain access, thus targeted subsidies or block tariffs would have
	to be strategically put in place \parencite{WB2016}.\\
	
	Pricing however, is less effective and carries more complexities for
	commercial purposes, which is the sector where most water is used. Sectors must
	also become more efficient, which is achievable through the design and adoption
	of new water saving technology, incentives and education. Climate Smart
	Agriculture (CSA) and Sustainable Agricultural Intensification (SAI) are some
	of
	the approaches currently available, and effectively allow farms to maintain or
	possibly increase returns while simultaneously reducing their water and energy
	use. This is a worthwhile pursuit as \parencite{Strong2020} indicate that the
	agricultural sector has the potential to be more sustainable at the lowest cost
	out of all of the different sectors of the global economy. There are also
	opportunities to alter behaviour and change water consumption through
	education.\\
	
	\paragraph{Reducing the impact of extremes, variability and uncertainty:}
	
	The goal will be to safeguard the world?s water supply through reaction to
	and reduction of natural disaster impacts and increased rainfall variability
	due
	to climate change. The challenge is not adequate available water but rather the
	distribution and stewardship of water. \parencite{Mekonnen2016} outline that
	there is enough water to satisfy all demand at a global level, this water is
	just unevenly distributed and so leaves water scarce and water dense areas.
	Large investment in technology and infrastructure will be required to mitigate
	the impacts of extreme events. Investment location and design must be sensibly
	thought-out, as these investments will be irreversible and expensive. These
	investments include seawalls, levees and dams for coastal cities, which could
	protect them from floods and storm surges. Upgrading or introducing
	hydro-meteorological and early-warning systems is another important investment.
	However, these forecasts will only be useful if they are understood by the
	users, so some investment into educating locals will be needed
	\parencite{WB2016}.\\
	
	Crop insurance is another useful response to the growing rainfall
	variability, especially in developing countries. It provides an incentive to
	invest in current technologies and crops of higher value, as this insurance
	removes the catastrophic risk of losing large investments if crops do prove
	unsuccessful \parencite{WB2016}.\\
	
	A single network must be used to supply water to consumers, as multiple water
	systems will be too costly to build and the network must have a single owner or
	monopolist that is subject to regulation to safeguard adequate access. However,
	the provider of the service has more knowledge than the regulator with regard
	to
	its own cost structure and efficiency level, meaning there is information
	asymmetry present which gives the provider an advantage and creates the
	possibility of inflated costs and inadequate services. Thus, policies and
	regulations need to identify these asymmetries, and ensure fair treatment of
	both the service provider, through a reasonable rate of return, and the
	customer, through affordable and quality services. An example of how poor water
	management systems can cost an economy is how South Africa loses around 7
	billion Rand a year due to non-revenue water in the form of pipe leakages
	\parencite{Madden2015}.\\
	
	\subsection{South Africa as a case study}
	Climate change poses threats to the South African agricultural sector, which
	accounts for a majority of the water usage in the economy, despite most
	agriculture being rainfed, and plays an important role in providing food,
	income
	and employment in the country. \\
	
	South Africa lies between latitudes 22 degrees  and 35 degrees South and
	longitudes 16 degrees to 33 degrees East. The country is split into nine
	provinces and has a wide variety of climates such as arid, temperate and
	subtropical. However, most of the agricultural activity takes place in arid and
	semi-arid regions \parencite{Waldner2017}. \\
	
	A population that is growing at around 2\% per year creates an increasing
	demand for food and water \parencite{Goldblatt2010}. The demand for water in
	South Africa is expected to increase by over 50\% between 2015 and 2030,
	leading
	to a 17\% deficit between the demand for water and the national supply, which
	is
	almost directly related by water mismanagement \parencite{Madden2015}.
	\parencite{Cammarano2020} suggest that climate change could increase the rate
	of
	poverty in South Africa by 2 to 3\%, however, adapting technologies focused
	around climate change could reduce poverty rates by 12 to 20\%.\\
	
	A unique blend of commercial and subsistence farming sustains around 70\% of
	the food supply, employment and incomes in the country
	\parencite{Cammarano2020}. The South African agricultural industry uses around
	60\% of the water that is used in the country every year, according to
	\parencite{Madden2015}, however, only around 30\% of the crops produced use
	irrigation \parencite{Waldner2017}.\\
	
	The two main crop types in South Africa by area planted are maize and wheat.
	Maize is used as a food staple for a majority of the population and is also
	essential to the animal sector of the agricultural sector as it is also used as
	feed. Around 75\% of maize produced in South Africa is planted in the Free
	State
	and Northwest provinces \parencite{Waldner2017}. \parencite{Cammarano2020}
	suggest that maize yields are likely to fall between 12 to 22\% as a result of
	climate change in South Africa. \parencite{Goldblatt2010} discusses how the
	annual maize production fluctuates in connection with weather conditions in the
	country, however, the average production of maize has remained stable in recent
	years. This is a worrying conclusion as the demand for food is expected to
	increase as a result of economic and population growth.\\  
	
	\parencite{Cammarano2020} outline the expected effect of climate change in
	South Africa as:
	\begin{itemize}
		\item Decreased rainfall in SA,
		\item Increased inter-annual weather variability,
		\item Increased probabilities of droughts,
		\item Increases in average temperatures and,
		\item Decrease in water due to changes in land use towards industrial and
		urban uses
	\end{itemize}
	
	The reliance of a majority of the country?s agriculture on rainfall and because
	of the inefficient water usage in the economy, South Africa is particularly
	susceptible to droughts. Especially since droughts are a common occurrence in
	Southern Africa and have negative impacts on agriculture, the natural
	environment and therefore, socio-economic well-being
	\parencite{Tirivarombo2018}.
	
	\subsection{Drought}
	Studies have shown that droughts and floods are the disaster types that have
	the widest reaching and worst impacts, especially on the African continent.
	Droughts are the disaster that is associated with the highest proportion of
	deaths related to natural disasters in Africa, with 46\% of deaths caused by
	natural hazards resulting from drought related conditions. This shocking number
	is only worsened by the statistic that 80\% of the people in Africa who were
	affected by a natural hazard, were impacted by drought \parencite{CRED2019}.
	Munish Re?s NatCatSERVICE indicates that between 1980 and 2016 floods, droughts
	and storms were responsible for 90\% of disaster events and 80\% of economic
	losses during the period \parencite{Cashman2018}.\\
	
	There are four different classifications for droughts. Meteorological drought
	is when there is a shortage of rainfall leading to a shortage of water. 
	Agricultural drought occurs when there is insufficient water availability or
	groundwater to sustain normal farming activity. Hydrological drought is related
	to insufficient surface and subsurface water levels based on a statistical
	average. Finally, socio-economic drought occurs when economic goods are in
	short
	supply because of weather-related shocks \parencite{UNDRR2015}.\\
	
	Drought disasters are some of the worst of all disaster types, especially in
	Africa, and threaten the livelihood and development of rural and urban
	populations. Droughts occur over longer periods of time than most other natural
	disasters. Droughts are not restricted to or predictably in certain geographic
	areas such as fault lines or river basins in the way earthquakes or floods are
	\parencite{UNDRR2015}.\\
	
	An outline of the issues around analysing droughts, according to
	\parencite{Tirivarombo2018}, is:
	\begin{itemize}
		\item Region specific factors play an important role in the socio-economic
		impacts of droughts
		\item Cumulative effects make the start and end harder to determine
		\item Drought develop slowly
		\item Only recognised when people and the environment
		start to feel the impacts
	\end{itemize}

	Although there are several challenges to analysing droughts and their impacts, the severity of these impacts can be avoided with good monitoring systems and an improved understanding of their characteristics. Thus, good early warning systems and drought assessment systems are essential in decision making around drought management policies \parencite{Tirivarombo2018}.
	
	\subsection{Finance, insurance and climate change}
	\subsubsection{Roles of insurance finance}
	The insurance industry plays an important role in generating economic growth by allowing the transfer of risk in order to minimise risks from unexpected future events. In this way, insurance facilitates investment in actions that are perceived to be risky. This is appropriate in the context of climate change where there are a lot of environmental and economic factors, which are not always well understood, that influence the expected losses arising from weather variability and increasing temperatures. An example of this, farmers in Lesotho have been able to take out crop insurance against drought risks, which allows them to expand their businesses \parencite{Cashman2018}. \\
	
	Governments play a role in providing protection and services in the form of employment, illness and natural disaster cover \parencite{Cashman2018}. Countries may deem it worthy to invest in infrastructure and policies that help protect against rainfall variability, if rainfall variability is predictable in those regions \parencite{Burby2006}.  
	
	\subsubsection{Risks}
	Although these policies will deliver relief in the short-term, they can also create problems associated with moral hazard. For example, insurance schemes that help protect against droughts and floods may encourage behaviour that increases exposure to the potential perils \parencite{Burby2006}. \\
	
	Other risks that climate change poses to the financial sector include over-valued companies, physical risks associated with climate change and transition risks as a result of climate mitigation and changing investor preferences \parencite{Farid2016}. \\
	
	Although physical risks are generally well understood, there is room for more quantification of the potential financial threats they pose. Physical risks are generally associated with unexpected increases in costs and frequencies of climate events. These are risks both to insurance liabilities from increased claim amounts and numbers, but also to asset values as investments in property may be damaged by a natural disaster. These risks may occur simultaneously depending on the business conducted by the insurer \parencite{Farid2016}.\\
	
	Transition risks arise from the loss of portfolio values due to climate mitigation or changes in investor and consumer preferences towards greener products and technologies. There is a lot more uncertainty around these risks and transition risks also carry over into the rest of the financial sector \parencite{Farid2016}.\\
	
	\subsubsection{Focus on the financial system}
	Many investors do not take climate factors into consideration when valuing their portfolios or companies. There is a risk that some companies may be overvalued due to holding ?stranded assets? and that, due to the long-term nature of these risks, investors are unwilling to adopt these risks into their portfolio management decisions. The issue here is that the ?carbon content? of securities could create unseen correlations in the market that could lead to large market fluctuations and losses of portfolio value \parencite{Farid2016}.\\
	
	Severe falls in GDP or the financial sector will put many governments in fiscal stress in the long term. This poses an increased threat as climate change is linked to increases in the occurrence of extreme weather events and natural disasters. These will in turn require significant expenses and management responses, this will only compound the fiscal stress placed on the government, thus creating a need to produce estimates of the extent of these budget pressures, which has yet to be convincingly explored \parencite{Kompas2018}.
	
	\subsection{GDP as a measrue of economic activity}
	The Gross Domestic Product is the market value of all final goods and services produced within a geographical entity within a given period of time. Although GDP has the advantage of being a simple measure, it does not adequately account for human, social and environmental welfare as well as ignoring non-market transactions such as at home work. It does however, adequately account for levels of employment, tax revenues and subsidies paid in the economy \parencite{Schepelmann2009}.\\
	
	GDP is a measure of production in terms of the price of economic goods. It is used as a measure of wealth and standards of living within a country \parencite{Schepelmann2009}. There have been several composite indices that have attempted to replace GDP with a wider view of what wealth is than just monetary wealth as assumed by GDP. GDP is seen as an imperfect measure of development and social well being and this is why there have been attempts at indices aimed at replacing GDP. However, these attempts at defining wealth produce inconsistencies and subjectivity to the theory backing the usage of these composite indicators. GDP is also well suited to measuring the goals and values of a capitalist society, that is, the creation of monetary wealth \parencite{Felice2016}.
	
	
	\newpage
	
	\section{Model}
	\subsection{Climate variables}
	Precipitation and temperature are good indicators of drought, as the literature
	indicates that climate events and droughts are correlated. These indicators can
	then be converted to drought indices which indicate the duration, magnitude,
	occurrence and intensity of a drought \parencite{Zargar2011}. Indices used to
	measure drought can be produced from both a single input variable and a mixture
	of hydrological variables \parencite{Hao2015}, those produced from hydrological
	variables are thought to provide results which are more certain, although the
	most appropriate variables do depend on the type of drought and the situation
	being addressed.
	
	\subsection{Drought indices}
	There are many drought indices that exist, such as the Standardised
	Precipitation Evapotranspiration Index (SPEI), the Weighted Anomaly
	Standardised
	Precipitation Index (WASP), the Rainfall Variability Index and the Standardised
	Soil Moisture Anomaly Index. However, the drought indices most universally used
	are the Standardised Precipitation Index \parencite{McKee1993} and the Palmer
	Drought Severity Index \parencite{Palmer1968}, with the SPI being the most
	popular drought index \parencite{Karabulut2015}. PDSI has precipitation,
	runoff,
	moisture supply and evaporation as input variables, while the SPI is based only
	on precipitation \parencite{VicenteSerrano2010}.
	
	\subsubsection{SPI vs PDSI}
	A drought index is multi-scalar when it is capable of accounting for the
	spatial scale and the various timescales at which droughts occur
	\parencite{Guttman1998} and since the SPI accounts for cumulative precipitation
	deficits at various spatial and temporal scales it is multi-scalar in nature
	\parencite{Hou2007}. The SPI can monitor the three types of drought, namely
	meteorological, agricultural and hydrological, because it is able to detect
	drought at various timescales because it possesses temporal flexibility. This
	is
	one of the major advantages the SPI has over other indices. PDSI however, is
	not
	able to detect drought at various timescales and thus is not multi-scalar in
	nature. Furthermore, PDSI requires various parameters to determine the index.
	Thus, not only is the SPI more comparable across regions with different
	climates
	than PSDI, but is also less complex to calculate as it only requires
	precipitation as a parameter. There are concerns however about the
	effectiveness
	of the SPI when change in drought associated with climate change is being
	measured. These concerns are due to the fact that SPI does not deal with
	changes
	in evapotranspiration.
	
	\subsubsection{Overview of SPI}
	\parencite{McKee1993} indicates that the SPI was designed to quantify observed
	precipitation as a standardised departure from a selected probability
	distribution function that models the precipitation data, where the
	precipitation data are typically fitted to a gamma distribution and then
	transformed to a normal distribution. Thus, the SPI values can be interpreted
	as
	the number of standard deviations by which the observed precipitation deviates
	from the long-term mean for a normally distributed random variable, as shown
	below. So, because the SPI is based on normalised data it is spatially constant
	allowing drought to be assessed in various regions \parencite{Guttman1998}. As
	mentioned before the SPI necessitates a single input of the long-term
	precipitation, which makes it less complicated when compared to other drought
	indices. The SPI value can then be used to compare and define drought
	conditions
	in different areas, and the index gives a reliable estimate of the severity,
	magnitude and spatial range of droughts. When computing the SPI, it is assumed
	that rainfall variability initiates droughts but that factors such as
	temperature are stationary, meaning that they are unchanged over time
	\parencite{VicenteSerrano2010}. A positive SPI value indicates that the
	precipitation value is higher than the long-term mean and conversely, a
	negative
	SPI value indicates that the precipitation value is lower than the long-term
	mean. The SPI output values range from -2 to +2, where -2 is extremely dry and
	+2 is extremely wet.\\
	
	$SPI=\frac{x_{i}-\bar{x}}{\sigma}$ \\
	
	Where $x_{i}$ represents the precipitation over the chosen period during year
	$i$, where $\bar{x}$ represents the long-term mean precipitation and $\sigma$
	is
	standard deviation over the chosen period.
	
	\subsection{Data}
	\subsubsection{Climate Data}
	To remedy sparse climate records over the region, this study uses the CRU TS
	v4.04 from the Climate Research Unit of the University of East Anglia
	\parencite{Harris2020a}. The data is a constructed 0.5\degree grid of monthly
	time series starting in 1901 and running through to 2019. The data contains
	several climate variables, of which this study will use the mean monthly
	temperature and precipitation.\\
	
	The data set is created by using the monthly average values in a base period
	of 1961 to 1990 to create a set of anomalies for observations over the
	remaining
	period by subtracting this mean from the observed data for each month. The
	gridded data set is then created through interpolation using the Angular
	Distance Weighting technique\parencite{Harris2020}.\\
	
	The choice to use this data set was checked by comparing the data values to
	ground weather station observations from all over South Africa using a data set
	published by the University of Cape Town \parencite{CIP}. The study tries to
	use
	observations at airports wherever available, to impose a basic form of quality
	control for the observations used. The study aims to find a $R^2 > 0.5$ between
	the CRU data and selected ground station observations
	\parencite{Tirivarombo2018}.
	
	\subsubsection{GDP data}
	The agricultural GDP data is derived from data on the GDP growth rate in South
	Africa and the contribution of the agricultural sector to the South African GDP
	over the same period \parencite{WorldBank2019} \parencite{WorldBank2019a}.
	
	\subsubsection{Crop specific data}
	The research uses total production of maize/corn and wheat in South Africa as
	well as the crop yields of these two sample crops \parencite{PDS2020}.
	
	\subsection{Methodology} 
	This research paper will look to calculate the SPI drought index over the
	period 1901 to 2019 at different time periods (1 month, 6 month and 12 month)
	\parencite{Tirivarombo2018}. These index values will then be compared to known
	events to discuss how accurately the SPI captures drought data during known
	drought periods based on news reporting of these events. Two events that will
	be
	highlighted are the 1904 and the 2015/2016 droughts in the Western Cape
	province.
	
	The SPI drought severity values will then be compared to Agricultural GDP
	growth and crop yields to check for correlation at a general level. To create
	further depth to the study the correlations between drought severity and the
	two
	mentioned variables will be compared between wetter years and drier years in
	order to further investigate the relationships. This methodology of using the
	same country data but for different states under investigation is used in a
	paper investigating the non-linear relationship of climate change and the
	economy \parencite{Burke2015}.
	
	\newpage
	
	\printbibliography
	
	\newpage{}
	
	\section*{Appendix\addcontentsline{toc}{section}{Appendix}}
	
	Additional code, output or data here. 
\end{document}


