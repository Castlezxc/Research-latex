%% LyX 2.2.2 created this file.  For more info, see http://www.lyx.org/.
%% Do not edit unless you really know what you are doing.
\documentclass[english]{article}
\usepackage[T1]{fontenc}
\usepackage[latin9]{inputenc}
\usepackage[a4paper]{geometry}
%\geometry{verbose,tmargin=3cm,bmargin=2.5cm,lmargin=2.5cm,rmargin=2.5cm}
\usepackage{color}
\definecolor{note_fontcolor}{rgb}{0.800781, 0.800781, 0.800781}
\usepackage{array}
\usepackage{verbatim}
\usepackage{rotfloat}
\usepackage{url}
\usepackage{amstext}
\usepackage{amsthm}
\usepackage{graphicx}
\usepackage{setspace}
\usepackage[backend=biber, style=apa]{biblatex}
%\bibliography{refs}
\addbibresource{refs.bib}
\usepackage{csquotes}
\usepackage{gensymb}

%%%%%%%%%%%%%%%%%%%%%%%%%%%%%% LyX specific LaTeX commands.
%% Because html converters don't know tabularnewline
\providecommand{\tabularnewline}{\\}
%% The greyedout annotation environment
\newenvironment{lyxgreyedout}
{\textcolor{note_fontcolor}\bgroup\ignorespaces}
{\ignorespacesafterend\egroup}
%%%%%%%%%%%%%%%%%%%%%%%%%%%%%% User specified LaTeX commands.
\usepackage{amsmath,epsfig,pstricks,pst-text,pst-node,pst-plot,amssymb,wrapfig,threeparttable,rotating}

\makeatother

\usepackage{babel}

\begin{document}

\title{Investigating the links between weather variability and agricultural GDP}

\author{Kevin Hendriks 17043779\\ 
	Danae Pavlou 17021856\vspace{0.5cm}\\
NPN 780 Research Report\vspace{2cm}\\
Submitted in partial fulfillment of the degree BSc(Hons) Actuarial Sciences\vspace{0.5cm}\\
Supervisor: S Hossain\vspace{2cm}\\
Department of Actuarial Sciences, University of Pretoria\vspace{0.5cm}\\
\includegraphics[width=5cm]{UPlogohighres}\vspace{0.5cm}\\
September 2020}

\date{}

\maketitle
\newpage{}
\begin{abstract}
Short summary of the research proposed. This will be a two to three
paragraphs long and aims to fully describe the content and contributions
of the research report.

\newpage{}
\end{abstract}

\section*{Declaration}

I, \emph{Kevin Ryan Hendriks and Danae Pavlou}, declare that this essay, submitted in
partial fulfilment of the degree \emph{BSc(Hons) Actuarial Sciences} at
the University of Pretoria, is my own work and has not been previously
submitted at this or any other tertiary institution. 

\vspace{1cm}

\_\_\_\_\_\_\_\_\_\_\_\_\_\_\_\_\_\_\_\_\_\_\_\_\_\_\_\_\_

\emph{Kevin Ryan Hendriks}

\vspace{1cm}

\_\_\_\_\_\_\_\_\_\_\_\_\_\_\_\_\_\_\_\_\_\_\_\_\_\_\_\_\_

\emph{Danae Pavlou}

\vspace{1cm}

\_\_\_\_\_\_\_\_\_\_\_\_\_\_\_\_\_\_\_\_\_\_\_\_\_\_\_\_\_

\emph{Saqib Hossain}

\vspace{1cm}

\_\_\_\_\_\_\_\_\_\_\_\_\_\_\_\_\_\_\_\_\_\_\_\_\_\_\_\_\_

Date

\newpage{}

\tableofcontents{}

\listoffigures

\listoftables

\newpage{}

\section{Introduction}
%(Not sure what the difference between an introduction and background is then??)
There is a gap in the coverage of climate related financial products around  less shocking, yet still devastating, climate related disasters, specifically drought. Existing financial products in the climate sphere often deal with natural disasters that produce sudden and obvious claim events such as earthquakes or cyclones. 

	\subsection{How do we address climate change?}
	Before climate change can be addressed through national and international policies, the current status of climate change needs to be established. This has been done in many research papers and is a major focus of several international organizations. There are two different categories of policies around climate:
	\begin{itemize}
		\item Policies that try to address the changing environment, try to slow down / reduce climate change. These policies are focused around limiting human activities that contribute towards increasing the amount of carbon in the atmosphere.
		\item Policies that look to protect communities, businesses and households from the negative impacts of climate. These look at changes in the environment and how people interact with these changes and are likely to be affected by these changes. The insurance industry is best suited to impact this side of the research question as, by nature, it is a risk-averse, defensive strategy.
	\end{itemize}
	
	The first looks to fight against future climate change where the second looks to defend vulnerable people against the current and future impacts of climate change. 

	In order to do this you need to establish what the future climate trends are likely to be, but it is also important to establish the effect these trends are likely to have. Fighting an effective battle against climate change requires both of these approaches to succeed and currently there is a lack of efforts with the aim of the second bullet point above.

\newpage

\section{Problem Statement}
The ideal for governments and financial institutions is to have methods of identifying which human activities contribute to climate change and by how much, then to be able to identify the climate events that are relevant to an economy and communities. Then to extend that to quantify the financial impact on the economy and various communities. Finally, to have financial products Investment projects should be aimed at preventing the loss of life and livelihoods in the case of a natural disaster or are capable of identifying / predicting the impacts of non-preventable changes in the climate, such as increasing temperatures. These models, policies and financial products are available to everyone and thus the climate related risks are spread globally.\\

However currently there are few early warning systems and few financial products that focus on climate change itself rather than on natural disasters as risks. The focus of funding is on humanitarian aid after a disaster event and not on projects that aid in preventing and coping with natural disasters. Existing models that attempt to quantify the impacts of climate change on various economies are difficult to access and require special expertise in coding, economics and statistics to be able to operate, usually requiring teams to adequately cover these fields. Due to the complex nature of these models most poorer regions of the world do not have access to the expertise or technology in order to run the appropriate analysis.\\

In order to try to bring what is currently available and being implemented closer to the ideal we need to explore the effects of the less violent natural phenomena as an important point that is not sufficiently addressed by the insurance or financial sector, currently. To progress in this regard, it is important to find a focal point within the many factors of climate change that can be analysed in a quantitative way. It is important to make the study as relevant as possible to regions of the world that have the most to gain from systematic improvements when it comes to handling climate change. This means that areas that have little to no climate analysis combined with large exposures to climate risk should be where the most urgency is directed when planning for future climate change mitigation.\\

In order to keep climate analysis as relevant as possible to the real world, the quantifiable measures that are used to construct financial products addressing climate change should not be abstract. These measures should be easily translatable into the real economy and therefore comparable to other financial risks. You want these quantifiable measures to be used in standard risk analyses where both the likelihood of a risk event occurring as well as the financial impact of a risk event occurring are important in decisions for further action by governments and financial institutions. The aim of being understandable and relevant to a financial audience as well as comparable to other financial risks is to encourage action in this space. 
 
\newpage

\section{Purpose statement}
The purpose of this research project is to investigate the relationship between agricultural GDP and drought severity in South Africa. 

\begin{itemize}
	\item The global region where the greatest impact from the focus of this study is possible is Africa, seeing as water is widely mismanaged and African communities are highly exposed to variance in weather conditions. 
	\item South Africa was chosen as a case study for the African continent as it has the most data coverage of the variables in this study. 
	\item GDP is chosen as a financially significant measure of the impact of climate change. 
	\item Agricultural GDP is chosen as this economic sector is most directly influenced by drought events. 
	\item Droughts represent a measurable event that is related to weather variability and is the link within the study to climate change.
\end{itemize}

Water scarcity is the point at which the two focuses of the study meet, those being the financial sector and climate change. Water scarcity is defined in terms of the supply vs the demand for water. Demand for water is driven by economic factors whereas the supply of water is an environmental factor, this provides the ideal point of departure for the research project. 

\section{Research objectives}
\begin{itemize}
	\item Isolate Agricultural GDP from the general GDP statistics 
	\item Collect measurements of drought severity
	\begin{itemize}
		\item Research various measurements of drought severity and pros and cons of a few different methods.
	\end{itemize}
	\item Identify a measure of water scarcity
	\item Check the collinearity of the measurement of water scarcity and drought severity
	\item Perform linear regression on agricultural GDP and the identified measurement of drought severity.
\end{itemize}
	 
\newpage

\section{Literature Review}

	\subsection{Climate change}
	Climate change is a broad field that covers a lot of interlinked factors and their effect on the globe. Climate change is change in the environment related to the temperature on the earth, the weather patterns around the globe and how they are changing \parencite{Pielke2004}. 

	\subsubsection{Human involvement in the process}
	There is large-scale consensus that a host of human activities worsen either the speed of climate change or the severity of the effect it has on human activity \parencite{Min2011}. Several organisations have published documents that promise support for climate change policies and set out schemes to address the issues it causes. These include the Paris Agreement from the IMF \parencite{Farid2016}, The Sustainable Development Goals from the UN \parencite{SDG2015}, the Sendai Framework from the UN \parencite{Sendai2015}, and the Stern Review from the Treasury of Great Britain \parencite{Stern2007}.\\

	\subsubsection{Environmental changes}
	Increasing global temperatures are linked to an increase in the frequency of natural hazards \parencite{CRED2019}. These hazards range from cyclones and storms, floods and droughts to earthquakes and volcanic activity. The literature shows varying levels of support for the link between climate change and these various hazards, but there are strong lines drawn to connect climate change and weather variability \parencite{Burke2015}. This may seem insignificant to humans with all our buildings and air-conditioning, but these changes in how hot or cold the earth is at certain times of the year fundamentally changes how we survive and how our economies run.

	\subsection{Humanity's climate footprint}
	The biggest problem is the way that human activity depletes natural resources and puts out gases, mostly CO2, into the atmosphere that change the way the earth handles heat from outside our atmosphere \parencite{Solomon2009}. On top of this, the normal activities of plants and animals in recycling these gases is diminished due to destruction of natural habitats and with them, the Earth's ability to regulate its temperature. 

	\subsubsection{The results of increasing CO2 emissions and increasing temperatures}
	The result of all this gas in the atmosphere is that temperatures will rise \parencite{Solomon2009}. This changes the flow of wind on the surface and in the atmosphere of the planet as wind is a movement of hot and cold air. If wind patterns change then weather patterns will also change. Most of the natural environment is in a fine balance and when the weather changes drastically, then so too will the natural environment. The change in temperatures also affects the behaviour of the ocean, which also plays a large role in dictating weather patterns, especially storms \parencite{Dasgupta2009}.\\

	All trends so far point to a greater variability in rainfall and storms \parencite{Damania2017}. There will be more rainfall in some regions and less in others. The problem here is that it is not going to be a fair distribution and so dryer places will most likely receive even less water and wetter regions will face even wetter conditions. This means that it becomes harder to predict whether rain will come and how much rain is likely to fall if a downpour does occur.

	\subsection{Do changing weather and rainfall patterns affect humanity?}
	Rainfall affects human society on three levels. Firstly, we need water to survive: to drink ourselves, to grow our food and to feed our animals. Secondly, human society must manage the water that falls on and runs through the spaces we live and work in. If this flow of water is not managed, floods or droughts occur. Both events have costs to human life as well as economic activity \parencite{CRED2019}. Finally, we use water for almost all our activities, both private and industrial \parencite{Strong2020}. Globally, our businesses and economies cannot function if they do not have water to use in daily activities. Some of these activities are more water-intensive than others, with manufacturing or mining activities being very water intensive, yet a school or office cannot run if there is no access to water. The balance between the demand for water and the supply of it is water scarcity.

	\subsubsection{Economic implications}
	Several studies link increasing rainfall variability, water scarcity, and increasing temperatures to negative economic impacts. \parencite{Burke2015} and \parencite{Dell2012} identify that rainfall and temperature variability are highly correlated and adversely effect economic growth. \parencite{Brown2013} used the Weighted Anomaly Standardized Precipitation index, which takes measurements of  rainfall inconsistency at extremes, to measure water availability. The results show that excessive or insufficient rainfall causes GDP growth rates to decrease. \parencite{Barbier2004} pays attention to 112 developing countries, finding that an increase in the use of water tends to slow the growth rates of developing countries by exploring the link between GDP growth and water utilisation rates. \parencite{Barbier2004} states that the reason for this is an increase in water utilisation does not increase productivity gains enough to outweigh the greater social and environmental costs associated with the infrastructure and organisations needed to achieve greater water security.\\

	\parencite{Stern2007} and \parencite{WB2016} both highlight that climate change has a substantial impact on a number of factors, namely: growth, production, and poverty reduction. \parencite{Stern2007} suggests that the distribution and variability of water is where people will feel the most significant impact of climate change. \parencite{WB2016} shares this view, finding that the impact of climate change will be felt principally through the water cycle. 

	\subsubsection{Is water scarcity a global issue?}
	\parencite{Mekonnen2016} find that about 71\% of the world experiences water scarcity for at least 1 month in any given year. They conducted a study that aims to address the underestimation of the impact of water scarcity by previous modelling exercises. The study states that due to the variation of rainfall patterns, it is inappropriate to use annually aggregated data to estimate the number of people impacted by water scarcity. The findings from the study are that half a billion people in the world face severe water scarcity throughout the year.

	\subsection{Dealing with water scarcity}
	Thanks to globalization, the world deals with water scarcity through trade \parencite{Damania2017}. Countries that have high levels of water scarcity import goods that they would not be able to support with local water levels. This trading action is called an implicit trade of water and it occurs whenever water is involved in the production of a good. This includes agricultural goods, most forms of manufacturing -- from the motor industry to the textile industry -- and minerals and ores through mining activities.

	\subsubsection{If these mechanisms exist to deal with water scarcity, what is the problem?}
	On a global level, there is enough water to meet the demand for water, however, the distribution of water availability and rainfall paints a different picture within any given region \parencite{Mekonnen2016}. Countries have their own water scarcity issues and balance their demand and supply through the implicit water trade, however, these systems can become overwhelmed when there is a prolonged period where there is less rainfall in a region than its economic sectors and water reserves can cope with. This happens when the balance between the demand for water changes faster than the political and economic systems can keep up with. 

	\subsubsection{Intensifying water cycles}
	\parencite{Stern2007} and \parencite{WB2016} predict the water cycle will intensify as climate change alters water availability, and many areas will see increased severity of droughts and floods. Arid areas, like Southern Africa, are likely to have increased water scarcity and \parencite{Barrios2010} as well as \parencite{Brown2011} found that irregular rainfall can have a substantial impact on economic growth in sub-Saharan Africa.

	\subsubsection{Drought}
	Drought disasters are some of the worst of all disaster types, especially in Africa, and threaten the livelihood and development of rural and urban populations. Droughts occur over longer periods of time than most other natural disasters. Droughts are not restricted or predictably in certain geographic areas such as fault lines or river basins for earthquakes or floods. There are four different classifications for droughts. Meteorological drought is when there is a shortage of rainfall leading to a shortage of water.  Agricultural drought occurs when there is insufficient water availability or groundwater to sustain normal farming activity. Hydrological drought is related to insufficient surface and subsurface water levels based on a statistical average. Finally, socio-economic drought occurs when economic goods are in short supply because of weather-related shocks \parencite{UNDRR2015}.\\

	There are issues with estimating the impact of droughts in terms of human capital or economic losses as mortality related to droughts often occurs through avenues such as a lack of food or civil unrest \parencite{UNDRR2015}. This creates a shortage of data around the links of drought to its impacts on human society. \\

	Despite these challenges, studies have shown that droughts and floods are the disaster types that have the widest reaching and the worst impacts, especially on the African continent. The natural disaster that makes up the most deaths that are related to natural disasters is droughts. In Africa, 46\% of deaths caused by natural hazards result from drought related conditions. Drought makes up the highest proportion of the number of people that are affects by natural hazards. Of the people in Africa that experience the effect of natrual hazards, 80\% of them are drought-related \parencite{CRED2019}.\\

	With this as the state of the world, it can be said that addressing water scarcity with a focus on efficient agriculture and drought resilience should be the first point of order when making policy decisions about climate change \parencite{Strong2020}. 

	\subsubsection{Disincentives for investment}
	Despite several publications warning of the increasing risks posed by droughts, there is a disproportionate lack of investments in measures to reduce these risks \parencite{UNDRR2015}. The current situation creates concerns for investors and the poor disaster management in many developing regions is a disincentive for investment \parencite{Hall2014}.\\

	\parencite{Amerasinghe2017} highlights that although there are a number of international funds that are aimed at funding projects that bring technology and infrastructure to deal with climate change as well as disaster risk reduction, these funds often miss their mandates of helping those countries most in need. Firstly, much of the financial aid provided to poorer regions is in the form of humanitarian aid rather that in the form of risk reduction systems and infrastructure \parencite{UNDRR2015}. Secondly, application systems around becoming accredited and gaining access to many of these international funds are inconsistent and biased to countries with better political and financial systems in the first place. These countries have an advantage with access to knowledge around the terms of funding from these international funds and how to design investment policies that satisfy these terms \parencite{Amerasinghe2017}. 

	\subsection{How are these systems overwhelmed?}
	That is a complex question. There are several supply and demand side factors that have a role in this situation occurring. 

		\subsubsection{The treatment of water as an economic good}
		The water cycle is multidimensional and thus it makes it difficult to address issues of water scarcity within a single policy \parencite{Damania2020}. At the source, water faces problems associated with a public good. Water policies are required to protect water sources. At the next stage of the water cycle, investment is required to convert water into a private good using infrastructure such as pipes and reservoirs. However, water is also seen as a human right and thus a merit good and so enough water needs to be made available at inexpensive prices. These definitions are contradictory and so there is no single price that would concurrently be able to meet both objectives.\\ 

		The distortion of water prices leads to inefficient and wasteful use of the resource, inflicting higher costs on the economy, society, and the environment. The marginal value of water for different uses varies substantially, as the price paid by residential, industry and agriculture users differ \parencite{WB2016}. The price of water rarely reflects its economic value, or the costs associated with treating and transporting it. Thus, a price that is too low and allows cheap water access may encourage inefficient water use and waste, compromising sustainability. Prices that are too high will mean the poor will be unable to access a resource that is a basic human right. Therefore, water prices must be able to control demand and guarantee unbiased access.

		\subsubsection{The supply of water }
		A single network must be used to supply water to consumers, as multiple water systems will be too costly to build. The network must have a single owner or monopolist that is subject to regulation to safeguard adequate access. However, the provider of the service has more knowledge than the regulator regarding its own cost structure and efficiency level. This presents an information asymmetry, which gives the provider an advantage and creates the opportunity to inflate costs and provide inadequate services. Thus, policies and regulations need to identify these asymmetries, and ensure fair treatment of both the service provider, through a reasonable rate of return, and the customer, through affordable and quality services \parencite{Damania2020}.

		\subsubsection{The difficulties of isolating economic impacts}
		Approaches have been advanced that try to directly deduce whether rainfall variability and water availability have a statistically evident impact on economic growth. However, it is impossible to identify the precise impacts as well as to quantify these impacts because of the complex interactions within an economy. Much of the literature published focuses on how water variations and availability impact certain sectors of the economy and find that the economic indicators are negatively impacted by adverse rainfall events -- such as floods and droughts. If one sector faces water supply shocks, this can have a ripple effect throughout the economy. Thus, evaluating one sector alone will not provide substantial evidence of the impacts on aggregated measures of economic activity like GDP \parencite{Damania2020}.

		\subsubsection{Population and urbanization strains}
		The World Economic Forum highlighted that there is an increasing demand for freshwater which is influenced by increasing global population. This creates an issue as high levels of water scarcity are often found in areas with high population density \parencite{Mekonnen2016}. Africa faces population growth issues that, if disaster risks are not managed properly, will increase the proportion of the global population that regularly experiences natural hazards. The proportion of the global population in Africa is expected to grow to 26\% in 2050, from 13\% in 2000 \parencite{CRED2019}. The number of people in these regions is expected to increase, however, a more specific consequence of this trend is that the likelihood of urbanization in areas exposed to natural hazards will also be increased. An example of water scarcity impacting urbanization is the 2010 drought in Syria, which lead to the migration of unemployed farm workers into the cities. This movement of people and the burden on the cities also contributed to the civil war around that time \parencite{Hall2014}.\\

		Resource shortages in urban areas and increases in population density creates a higher exposure to any single disaster and sets up conditions that are ripe for civil conflict and disease. Another concern with urbanization in Africa is that over 60\% of Africa urban population live in slums or suffer from some form of deprivation that defines a slum. This means that these populations face a lack of clean water, sanitation facilities, durable housing, and sufficient living space. All these conditions set the perfect scene for disease outbreak or other losses of human life were a natural disaster to strike. The African continent experiences the highest frequency of epidemics, mostly due to the presence of malaria and HIV/AIDS in the region. However, poor hygiene practices and food insecurity heighten the impact of these biohazards \parencite{UNDRR2015}. 

		\subsubsection{Agricultural irrigation}
		The agricultural sector of developing nations usually makes up a large proportion of national income \parencite{Johnston1961}. The World Economic Forum includes the expansion of agricultural irrigation as a factor in increasing demand for freshwater \Parencite{Mekonnen2016}. The World Bank draws a connection between rainfall variability, specifically dry weather shocks, and increased deforestation and conversion of land into cropland \parencite{Damania2020}. The World Bank indicates that there is a lack of education around the disaster risk reduction in much of the developing world. This leads to avoidable land degradation and a reduction in farming efficiency, which erodes agricultural incomes but also negatively impacts these regions' exposure to weather shocks and their capability to recover from them \parencite{UNDRR2015}.\\

		Access to irrigation can create inappropriate crop choices. More water is generally allocated to the agricultural sector and it is often made available to the user at virtually no cost, leading to the overuse of water \parencite{Damania2017}. The opportunity cost of supplying water to agriculture, industry and residential sectors is often unrelated to the price paid for the water by sectors \parencite{Strong2020}. Water for agriculture is normally cheaper than for urban consumption, and although this can be explained, to an extent, by the difference in the product quality, the fundamental reason for this mismatched price is because the market does not use the economic value of water when allocating it. \parencite{Mekonnen2016} also find that areas with high levels of irrigation compared to the natural water availability also experience high levels of water scarcity. This can be seen in areas such as the Limpopo Basin in South Africa, where, despite being a river basin, the area creates water scarcity by having countercyclical water consumption patterns. These unsustainable water practices are expected to lead to dryer riverbeds and decreased groundwater. Overall, this increases the likely impacts when a drought does occur where the agricultural sector experiences reduced harvests and a loss of income to farmers.

	\subsection{Welfare impacts and the far-reaching implications for individual income}
	There is an increasing amount of evidence demonstrating that rainfall shocks can have long-term economic effects on human capital. According to \parencite{WB2016} there is a long-term effect on health, human capital, and education, especially in young children because of a lack of water or the having no other option but to use contaminated water. Through these factors, climate produced water shocks hinder development and poverty reduction. This proves to be true especially when rainfall shocks result in health impacts that adversely affect young children or when families are unable to invest in their children due to a lack of income \parencite{Almond2011}. A rainfall shock can significantly decrease a family's income when income depends on favourable rainfall patterns. This lack of income could cascade into foetuses and infants developing nutritional deficits which could have dire consequences in the long run. This could extend the cycle of poverty, as health concerns and malnutrition brought on by floods or droughts can impact education and thus future employment.\\

	Water scarcity and climate change can have a long-term effect on the health and economic potential of rural farmers residing in developing countries but also middle-class workers from middle-income countries. \parencite{Maccini2009} conducted a survey of Indonesian rural farmers and found that persons, especially women, who in their first year of life experienced rainfall shocks also achieved less education, earned lower incomes and married people who earned lower incomes. Interestingly, recent evidence demonstrates that these results could impact middle-class workers from middle-income countries. A study concerning middle class workers in Ecuador shows that if rainfall shocks and higher temperatures were experienced while still in the uterus, the results would see the child experience reduced formal sector earnings 20 to 60 years later \parencite{Fishman2015}.

	\subsection{Is there anything being done to combat these issues?}
	\parencite{UNDRR2015} speaks about several measures that are being undertaken to improve Disaster Risk Reduction, however, highlights that these systems are neither well integrated into the social systems within countries nor across country borders. There are several key issues that hinder the progress of Disaster Risk Reduction policies within the African continent. Most of the early warning systems in Africa are managed by external organisations, such as NGOs. There is a lack of initiatives to collect useful or comparable data on meteorological patterns or on vulnerability factors. This would be an issue in an effort to address any policy decisions, however, since weather related changes are poorly understood and disaster events are extreme events, this lack of data is a massive set back to addressing these challenges \parencite{Hall2014}. Finally, there are only a few specialised institutions set up to research and fund projects aimed at increasing preparedness and strengthening capacity to manage climate risks.\\
	 
	There is also a fundamental issue with how humans deal with disaster risk mitigation on a global scale and that is that investment in disaster risk reduction usually only happens reactively. That is to say that there is a pattern of investing in large-scale disaster prevention infrastructure only after a major event has already occurred \parencite{Hall2014}. Considering the magnified exposure to disaster risks in Africa as well as the potential post-disaster, humanitarian crises, work together with the above systematic concerns to set tremendous boundaries to sustainable development in the region \parencite{UNDRR2015}.Disaster Risk Reduction aims at reducing the economic and social strain of natural hazards



	\subsection{How do we Predict these challenges}
	Much of the literature that explores the effect of climate change on economic growth uses similar models, including Integrated Assessment Models and Computable General Equilibrium (CGE) models. The results of these models are that the effect of climate change on the global economy are meaningful, and that the negative consequences are felt worst in developing countries.  

		\subsubsection{Integrated Assessment Models and Computable General Equilibrium (CGE) models}
		Integrated Assessment Models that have tried to quantify the impacts of climate change work on aggregate modelling, sometimes only including one good produced and consumed in each economy. There are fewer models that represent a computable general equilibrium structure. CGE models are useful in reflecting more sector-specific impacts and differences in the economic structure.\\

		A computable general equilibrium model could alternatively be used to determine how water impacts the economy. When statistical information is lacking, CGE models project the consequences of hypothesised scenarios. They provide projections of the consequence of hypothesised scenarios in a stylized depiction of the economy, not predictions and forecasts.  The models are determined by computational constraints and data availability and are based on a series of assumptions. These models aim to provide clarity on the extent and direction of change and how adverse consequences could be alleviated or heightened by policies. 

		\subsubsection{Examples of IA Models that use CGE techniques}

			\paragraph{Shared Socioeconomic Pathways (SPPs) and CGE Models}
			\parencite{WB2016} suggests that water scarcity hinders growth and development, with the greatest water scarcity experienced in the Middle East, Central Asia, parts of South Asia and North Africa. The study exhibits how significant action is needed to prevent climate related water issues from having adverse implications for global economic growth. The model explores how water scarcity could constrain economic growth by comparing two extreme scenarios with SSP1 and SSP3. SSP1 represents positive outlook and SSP3 describes a world comprising of low adaptation, restricted economic growth, and high emissions. The Shared Socio-economic Pathways (SSPs) are narratives that describe stylised development scenarios and define changes in technology, lifestyles, demographics, policy, and the economy. Water supply projections from a range of hydrological models are integrated into a CGE which preforms simulations, roughly showing the inter-relation of economic sectors and how a shock in one sector could cascade through to the whole economy.

			\paragraph{ENVISAGE}
			\parencite{Roson2012} used an integrated assessment model (ENVISAGE) to show that the long run impacts of climate change are substantial, especially in poorer countries. They link a large part of GDP loss in North African and the Middle East to a decrease in labour productivity, followed by water scarcity. The model assesses the economic impacts of various climate factors, but cites water scarcity as the reason for the negative welfare effect on the Middle East and North Africa, which are expected to be -16.14\% by the end of the century. This model evaluates the general magnitude and distribution per region of the impacts, as well as the consequences of each impact on overall income and welfare variation. The ENVISAGE model is more sophisticated some previous models and uses improved parameter estimates. It uses a recursive dynamic approach, with adaptive expectations. This allows more climate change impacts to be considered. The model compares two different scenarios, one that ignores any climate change and one that accounts for climate change and its impacts.

			\paragraph{SSPs and RCPs}
			\parencite{Kompas2018} extended a CGE trade model to show that the effect of global warming, although varying by region and economic sectors, increases in the long run and are severely worse in poor, African and Asian nations. The model considers the many effects of increasing temperature on GDP growth and GDP levels for 139 countries. The model, which is fully disaggregated, was extended to account for the effect of various Shared Socioeconomic Pathways (SSPs) and Representative Concentration Pathways (RCPs) on global temperature. This approach predicts a 1 - 4\degree C increase in global temperatures. The GDP loss in these areas and all areas near the equator is the most significant. At a global warming of 3\degree C, the expected global potential loss is estimated at US\$9,593.71 billion at 2100. When global warming reaches 4\degree C losses are estimated to increase to US\$23,149.18 billion.  

		\subsubsection{Comparable Results}
		The results using the ENVISAGE model are comparable to findings by \parencite{Kompas2018}. ENVISAGE bases results on only 15 regions, which unavoidably averages outcomes, where Kompas, Van H and Nhu Che use a larger dimensional model, which accounts for 139 countries. \parencite{Roson2012} found that by the end of the century, an increase in global temperature of 4.79\degree C will mean parts of Asia lose 12.6\% of their GDP and the Middle East and North Africa lose 10.3\%.  \parencite{Kompas2018}  show that for a global temperature increase of 4\degree C, South East Asian countries can lose as much as 21\% of their GDP yearly and finds a GDP loss of 26.6\% yearly for developing countries in Africa.\\

		The fundamental finding is that the poor will suffer the most from the impacts of climate change. Another key takeaway is that water scarcity will be where the impact of climate change will most significantly be felt, which agrees with \parencite{Stern2007}.

		\subsubsection{Inconsistencies of results }
		There are studies that show that rainfall or water availability has no strong and significant effect on economic growth, which is contradicting when considering the robust micro econometric indication of the extensive effect of rainfall and water availability on important economic indicators. \parencite{Stern2007} considers how a change in climate could impact people, the environment and the potential for economic growth and development in various regions. It discerns that the effects across the global will be different, where some regions may benefit from small temperature rise. But if temperature increases rise above 2-3\degree C, which is anticipated in the future, most countries will feel the effect and global growth will suffer.\\

		It has been suggested that a reason for the discrepancy of results is due to the statistical approaches used, which place attention on how water availability or rainfall impacts economic growth at the national level. \parencite{Briant2010} showed that this is due to spatial averaging over large units, which disguises variability and explains why impacts of rainfall on GDP are found to be fragile.  If a part of a country is experiencing floods while the other is experiencing droughts, averaging over the country may result in normal rainfall, thus inadequately capturing the spatial variation in water availability, and the varying effects it has on a country. However, at disaggregated spatial scales the analysis shows that rainfall variability has a substantial impact on economic growth \parencite{Damania2019}. 

		\subsubsection{The limitations of CGE Models}
		When data is sparse the CGE approach is arguably the best available method, CGE models do however have limitations. They assume that it is easy to substitute factors, for example when a shock results in an input like water being reduced, it is assumed that a factor such as capital can replace the input and that this occurs at no loss of production. This is not the case in practice, as capital and water are not the same, and where substitution is possible it will not be perfect and will involve time and transition costs. CGE models struggle to deal with externalities as well as the key features of the hydrological cycle, which is vital when water is concerned. Clearly integrating water into a CGE proves difficult because of the complex nature of water and the absence of information on water use. Water is renewable and often reused, as the water cycle is iterative. Most CGE models overlook the complex traits of water. Thus, it is challenging to identify and quantify fully, the various complex relations between water and the economy, so CGE model results should be used with care.

	\subsection{The potential benefits of improving water management practices}
	\parencite{WB2016} concludes that  judicious water management can partially nullify the adverse growth impacts the model's fundamental discovery is that prudent water-management policies can greatly protect growth, increase wealth, and consequently increase resilience to climate stress. Efficient water pricing and the reallocation of water are two such policies. By introducing more efficient water prices, while still implementing policies that protect the poor, sufficient water can be conserved thus ensuring that enough water is available to meet basic needs. Secondly, water efficiency would be attained if water were used more productively within sectors and through the allocation of water to higher value usages. A restructuring of water rights and water governance mechanisms is needed. An abundance of CGE models show that efficient allocation can lead to meaningful benefits. \parencite{Hassan2011} showed that efficient water allocation in South Africa could enhance the agricultural GDP by approximately 4.5\% annually, further \parencite{Roson2017a} show that inefficient allocation results in estimated global losses of 6\% of GDP.\\

	Education programs around disaster risk reduction can lead to overall social benefits. \parencite{UNDRR2015} suggests that to improve education around environmental hazards, the general education systems in many poor areas would need to be improved. This may lead to a more educated population in general, including around gender equality and land use.\\ 

	The most realistic and significant improvements towards water sustainability and the SDG 6 are in the agricultural sector  \parencite{Strong2020}. There are several studies mentioned in this literature review that link rainfall variability and drought to negative economic impacts. Thus, there is scope for insurance and social safety nets to prevent much of these negative impacts by protecting the incomes of the agricultural sector \parencite{UNDRR2015}. There are a number of supply side approaches mentioned by \parencite{Damania2017}, but these are harder to implement without national buy-in and access to funding, as described by \parencite{Amerasinghe2017}, or without social awareness through disaster risk education and early warning systems \parencite{UNDRR2015}. Thus, there it is important for the private sector to step up and a the ability to classify droughts and to quantify appropriate financial actions based on the severity of their impacts on communities. 
	\newpage

\section{Model}

	\subsection{Research Methods}

	The main aim of the model will be to provide some estimated answer as to the economic impacts of climate change on South Africa. This paper will focus on water scarcity as droughts and floods are the significant disaster risks that the Sub-Sarahan region faces \parencite{CRED2019}.\\

	Institutions often underestimated the impacts of and their exposure to droughts, as can be seen in the impacts of the recent drought that South Africa has experienced and the nature of droughts in general \parencite{Damania2017}. One of the main contributing components to a drought is the rainfall patterns within a region. Countries like South Africa rely on a rainy season in order to meet agricultural needs for food and animal feed. Volatile rainfall patterns, in combination with inappropriate water management systems, can lead to worrying levels of exposure to the negative impacts of drought. There have been several studies that have tried to use spatial analysis to capture the rainfall patterns in specific areas and globally and map out when droughts and particularly wet periods have passed. Most of these studies have used the Delaware data to perform this analysis. This paper will not try to recreate these spatial models, but will rather accept their results and try to link these to economic impacts. The main factor that this paper will focus on as an indicator of the economic performance of a country will be Gross Domestic Product (GDP).\\ 

	The model will analyse the correlations between different indicators of rainfall patterns, such as precipitation, to the overall GDP of a region over time. The model will have some calibrating periods where there have been known periods of droughts, such as recently in South Africa. The data for this study will come from the Delaware data set, the Demographic and Health Surveys Program run by USAID, the EARTHSTAT database, the World Bank World Development Indicators dataset and the United Nations World Population Prospects 2019 data set.\\

	After establishing any correlations between drought indicators and economic indicators, the model will look to forecast these economic costs at least until 2030, based on the United Nations 2030 Agenda, under several different economic and climate scenarios. This scenario forecasting will require further research into the combinations of factors that would be appropriate to create feasible future scenarios.

	\subsection{Possible Problems}

	The possibility of insufficient data should not be an issue in the approach the model will take as all of the suggested datasets have data records over several decades. However, the data may be somewhat inappropriate, and issues around the granularity of the data may still exist. There may not be enough data to cover the rainfall in only one specific country as the data could be captured on a country by country level. However, this does not seem to be the case through a preliminary viewing of the data sets. Some other issues that the model will have to take into account will be the interference of other factors in the analysis of correlations. Examples of these external factors, external here referring to being outside of a drought or other factors of water scarcity, such as economic recessions. Further internal issues may arise in the analysis of rising temperatures and rainfall volatility as these two factors are closely linked and are both seasonal \parencite{Damania2017}.



	\section{Conclusion}

	There are many avenues to address water scarcity and, more broadly, climate change. There are several internationally
	run funds that look to address general and specific aspects of climate change, from disaster relief to adopting new technologies.
	Adequately addressing all aspects of climate change will require both public and private investment in climate infrastructure and
	solutions. A report by the World Resource Institute  \parencite{Strong2020} attempts to put a dollar figure on the private sector investment that would be needed to meet the SDG 6 laid out in the United Nations' 2030 Agenda. These investments would go towards infrastructure to combat access and storage of water. Another report, by the World Bank, advises on the cost of addressing the various shortfalls in private water management \parencite{Damania2017}. A useful suggestion from this book is the use of tradable water rights that are in the form of standard contracts. This paper sees a unique opportunity in South Africa to adopt this due to the strength of our financial sector and systems. Another supply-side issue is the water dropped during transportation in water infrastructure, such as water pipes. South Africa loses around 90 billion Rand per annum due to leaks in this infrastructure \parencite{Strong2020}. Both the public sector and the private sector would need to implement these approaches to combat the adverse effects of climate change adequately.

\newpage
\printbibliography

\newpage{}

\section*{Appendix\addcontentsline{toc}{section}{Appendix}}

Additional code, output or data here. 
\end{document}
