%% LyX 2.2.2 created this file.  For more info, see http://www.lyx.org/.
%% Do not edit unless you really know what you are doing.
\documentclass[english]{article}
\usepackage[T1]{fontenc}
\usepackage[latin9]{inputenc}
\usepackage[a4paper]{geometry}
\geometry{verbose,tmargin=3cm,bmargin=2.5cm,lmargin=2.5cm,rmargin=2.5cm}
\usepackage{color}
\definecolor{note_fontcolor}{rgb}{0.800781, 0.800781, 0.800781}
\usepackage{array}
\usepackage{verbatim}
\usepackage{rotfloat}
\usepackage{url}
\usepackage{amstext}
\usepackage{amsthm}
\usepackage{graphicx}
\usepackage{setspace}
\usepackage[backend=biber]{biblatex}
\addbibresource{BibliographyDatabase.bib}
\doublespacing

\makeatletter

%%%%%%%%%%%%%%%%%%%%%%%%%%%%%% LyX specific LaTeX commands.
%% Because html converters don't know tabularnewline
\providecommand{\tabularnewline}{\\}
%% The greyedout annotation environment
\newenvironment{lyxgreyedout}
  {\textcolor{note_fontcolor}\bgroup\ignorespaces}
  {\ignorespacesafterend\egroup}

%%%%%%%%%%%%%%%%%%%%%%%%%%%%%% Textclass specific LaTeX commands.
\theoremstyle{plain}
\newtheorem{thm}{\protect\theoremname}
\ifx\proof\undefined
\newenvironment{proof}[1][\protect\proofname]{\par
\normalfont\topsep6\p@\@plus6\p@\relax
\trivlist
\itemindent\parindent
\item[\hskip\labelsep\scshape #1]\ignorespaces
}{
\endtrivlist\@endpefalse
}
\providecommand{\proofname}{Proof}
\fi

%%%%%%%%%%%%%%%%%%%%%%%%%%%%%% User specified LaTeX commands.
\usepackage{amsmath,epsfig,pstricks,pst-text,pst-node,pst-plot,amssymb,wrapfig,threeparttable,rotating}

\makeatother

\usepackage{babel}
\providecommand{\theoremname}{Theorem}

\begin{document}

\title{Research report title (take note where capitals letters should be)}

\author{First Name and Surname Student Number\vspace{0.5cm}\\
WST795/STK795 Research Report\vspace{2cm}\\
Submitted in partial fulfillment of the degree BSc(Hons) Mathematical
Statistics \\
/ BCom(Hons) Mathematical Statistics / BCom(Hons) Statistics\vspace{0.5cm}\\
Supervisor(s): Initals Surname, Co-supervisor(s): Initials
Surname\vspace{2cm}\\
Department of Statistics, University of Pretoria\vspace{0.5cm}\\
\includegraphics[width=5cm]{UPlogohighres.jpg}\vspace{0.5cm}\\
Enter date of document here}

\date{}

\maketitle
\newpage{}
\begin{abstract}
Short summary of the research proposed. This should be a two to three
paragraphs long and should fully describe the content and contributions
of the research report.

\newpage{}
\end{abstract}

\section*{Declaration}

I, \emph{full students name, }declare that this essay, submitted in
partial fulfillment of the degree \emph{BSc(Hons) Mathematical Statistics
/ BCom(Hons) Mathematical Statistics / BCom(Hons) Statistics, }at
the University of Pretoria, is my own work and has not been previously
submitted at this or any other tertiary institution. 

\vspace{1cm}

\_\_\_\_\_\_\_\_\_\_\_\_\_\_\_\_\_\_\_\_\_\_\_\_\_\_\_\_\_

\emph{Insert Student's full name Here}

\vspace{1cm}

\_\_\_\_\_\_\_\_\_\_\_\_\_\_\_\_\_\_\_\_\_\_\_\_\_\_\_\_\_

\emph{Insert Supervisor(s) name(s) here}

\vspace{1cm}

\_\_\_\_\_\_\_\_\_\_\_\_\_\_\_\_\_\_\_\_\_\_\_\_\_\_\_\_\_

Date

\begin{comment}
Edit all italics with your details and sign on the solid lines once
the final document is printed and submitted. 
\end{comment}

\newpage{}

\section*{Acknowledgements}

Add acknowledgements here (not compulsory) e.g. bursaries received. 

\newpage{}

\tableofcontents{}

\listoffigures

\listoftables

\newpage{}

\section{Introduction}

The introduction should provide a detailed description of the topic
and the aim of the research report. In addition the literature review
should intertwine with this. It is important to always reference where
needed. All work from somewhere else requires a reference \cite{Anguelov2010}.
QUESTIONS/POINTS TO COVER:

-

-

-

-

-

Inline equation: $x=y$

Display equation: \[x=y\]

Numbered display equation: 
\begin{equation}
x=y\label{eq:afe,fjgd}
\end{equation}

Numbered equation: 
\begin{equation}
x=y+1
\end{equation}

Equation array: 

\begin{eqnarray*}
x & = & y+2+3\\
 & = & y+5.
\end{eqnarray*}


\section{Background Theory}

In Figure \ref{fig:UPlogo} and in Figure \ref{fig:Caption} we see
the UP logo. %

\begin{figure}[h]
\centering
\includegraphics[width=6cm]{UPlogohighres.jpg}
\caption{A graphic of the UP logo\label{fig:UPlogo}}
\end{figure}

- What should be in a caption? 

- How should I get my graphics?

\begin{figure}
\includegraphics[width=6cm]{UPlogohighres.jpg}
\caption{Caption \label{fig:Caption}}
\end{figure}
 The theory of the topic should be thoroughly discussed in this section.
The student must show their proficiency on the topic as well as additional
insight. This section may be separated into a few sections as \textit{necessary}.
Note that the theory of the topic is the main contribution and the
report should illustrate this. In Theorem \ref{thm:first} we discuss.....
$f(x)=\left\{ \begin{array}{cc}
1 & \textrm{\textrm{if }}x<2\\
2 & \textrm{if }x\ge2
\end{array}\right.$
\begin{thm}
\label{thm:fbkndfblksnfbkn}fbkndfblksnfbkn 

jbj fbfjk b bfjf $f(x)=\left\{ \begin{array}{cc}
1 & \textrm{\textrm{if }}x<2\\
2 & \textrm{if }x\ge2
\end{array}\right.$

\end{thm}
\begin{proof}
blksnfblkdfsnb
\end{proof}
m,fgkrjtbnel
\begin{thm}
\label{thm:first} aurbyveurybverybyruwei 
\end{thm}
\begin{proof}
vnitunrivuptnriuwvtnoiutnreuvnt
\end{proof}

\section{Application}

\begin{sidewaystable}
\begin{centering}
\begin{tabular}{|>{\centering}p{3cm}|c|c|c|>{\centering}p{3cm}|}
\multicolumn{1}{>{\centering}p{3cm}|}{} &  &  &  & \tabularnewline
klvunl ruvynryrinfsfgng hdfhfgjf gjfhjghjhk jhkghjkh dfhdj ghkjfgj
djdj dghjhhjghj & aevirue$x=y$n oiurlevniq & ermwovmi;q & aevpimqw & ovmi;\tabularnewline
 &  &  &  & \tabularnewline
 &  &  &  & \tabularnewline
 &  &  &  & \tabularnewline
\end{tabular}
\par\end{centering}
\caption{rvybruvybrv \label{tab:test}}

\end{sidewaystable}

\begin{table}[h]
\begin{centering}
\begin{tabular}{|>{\centering}p{4cm}|>{\centering}p{4cm}|>{\centering}p{4cm}|}
\hline 
\multicolumn{1}{|>{\centering}p{4cm}||}{} & \multicolumn{1}{>{\centering}p{4cm}||}{Column 1} & Column 2\tabularnewline
\hline 
\hline 
Row 1 &  & \tabularnewline
\hline 
Row 2 &  & \tabularnewline
\hline 
\end{tabular}
\par\end{centering}
\caption{Example \label{tab:Example}}
\end{table}
 The application should be presented in this section. Code should
be included in an appendix as well as additional output if needed. 

\section{Conclusion}

The conclusion should summarise what was done in the research report.
It should also provide shortfalls of the research and recommendations
on what could be investigated in future. This section should be an
honest summary of the research.


\newpage

\printbibliography

\newpage{}

\section*{Appendix\addcontentsline{toc}{section}{Appendix}}

Include any additional code, output or data here. 
\end{document}
